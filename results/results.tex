\subsection{Data and Input Parameters Analysis}

[cite_start]The validity of the Agent-Based Model (ABM) was established through the rigorous empirical grounding of its initialization parameters, derived from legislative documents, key informant interviews, and financial data from the Municipality of Bacolod[cite: 1, 2]. [cite_start]The simulation environment was strictly bounded by a hard Annual Budget Cap of \textpeso 1,500,000 (approx. \textpeso 375,000 per quarter), a constraint that created a ``Personnel Trap'' where the high cost of enforcement (approx. \textpeso 24,000--\textpeso 30,000 per enforcer) frequently crowded out funding for education and incentives[cite: 8, 14, 15].

[cite_start]As detailed in Table \ref{tab:barangay_demographics}, the municipality exhibited profound heterogeneity, preventing the success of a monolithic ``one-size-fits-all'' policy[cite: 50, 51]. [cite_start]Barangay Poblacion, termed the ``Urban Resource Trap,'' possessed the largest budget (\textpeso 200,000) but suffered from the lowest per-capita spending power due to its massive population, forcing a legislative lock of 60\% on enforcement to maintain order[cite: 22, 23]. [cite_start]Conversely, Barangay Binuni represented a ``Wealthy Anomaly'' with a fiscal surplus that allowed for a balanced strategy, while Barangays Babalaya and Demologan exemplified the ``Poverty Trap,'' where nearly 90\% of meager funds were consumed by mandatory personnel costs[cite: 26, 42, 44].

\begin{table}[htbp]
    \centering
    \caption{Barangay Demographic and Financial Initialization Parameters}
    \label{tab:barangay_demographics}
    \small 
    \begin{tabular}{l r r c c c c}
        \toprule
        & & {Annual} & {Initial} & \multicolumn{3}{c}{{Income Profile (\%)}} \\
        \cmidrule(lr){5-7}
        {Barangay} & {Households} & {Budget (\textpeso)} & {Compliance} & {Low} & {Mid} & {High} \\
        \midrule
        Brgy Poblacion    & 1,534 & 200,000 & 2\%  & 70 & 25 & 5 \\
        Brgy Liangan East & 608   & 30,000  & 14\% & 40 & 40 & 20 \\
        Brgy Ezperanza    & 574   & 90,000  & 14\% & 20 & 50 & 30 \\
        Brgy Binuni       & 507   & 126,370 & 15\% & 50 & 30 & 20 \\
        Brgy Demologan    & 463   & 21,000  & 11\% & 80 & 15 & 5 \\
        Brgy Babalaya     & 171   & 15,000  & 14\% & 80 & 10 & 10 \\
        Brgy Mati         & 165   & 80,000  & 11\% & 90 & 5  & 5 \\
        \bottomrule
    \end{tabular}
    \vspace{1ex}
    
    \raggedright
    \footnotesize
    \textit{Note: Income profiles derived from Appendix B interviews. [cite_start]Low Income agents have higher price sensitivity ($\gamma$) to fines[cite: 21].}
\end{table}

The simulation further incorporated distinct psychosocial profiles for each barangay using Theory of Planned Behavior (TPB) weights, as summarized in Table \ref{tab:behavioral_parameters}. [cite_start]The economic disparity between agrarian zones and commercial centers fundamentally altered the marginal utility of money; for instance, the high concentration of low-income households in Mati and Demologan resulted in high price sensitivity ($\gamma$), making uniform fines disproportionately punitive[cite: 53, 55].

[cite_start]Behavioral nuances were also encoded: Barangay Poblacion exhibited an ``Urban Isolation'' profile with low Social Norm weights ($w_{sn}=0.20$), indicating that community pressure was ineffective in high-density areas[cite: 69, 70]. [cite_start]In contrast, agricultural areas like Barangay Mati showed high habit decay rates ($0.10$), validating the need for continuous intervention to prevent residents from reverting to traditional practices[cite: 76].

\begin{table}[htbp]
    \centering
    \caption{Calibrated Behavioral Weights and Legislative Allocation Strategies}
    \label{tab:behavioral_parameters}
    \resizebox{\textwidth}{!}{ 
    \begin{tabular}{l c c c c c c c c}
        \toprule
        & \multicolumn{4}{c}{{Agent Behavior Parameters}} & & \multicolumn{3}{c}{{Budget Allocation (\%)}} \\
        \cmidrule(lr){2-5} \cmidrule(lr){7-9}
        {Barangay} & {Attitude ($w_a$)} & {Norms ($w_{sn}$)} & {Effort ($C_e$)} & {Decay ($\gamma$)} & & {Enf} & {Inc} & {IEC} \\
        \midrule
        Binuni       & 0.65 & 0.80 & 0.64 & 0.02 & & 40 & 40 & 20 \\
        Ezperanza    & 0.40 & 0.70 & 0.55 & 0.03 & & 50 & 30 & 20 \\
        Babalaya     & 0.60 & 0.90 & 0.62 & 0.05 & & 90 & 5  & 5 \\
        Liangan East & 0.65 & 0.60 & 0.58 & 0.04 & & 25 & 65 & 10 \\
        Poblacion    & 0.55 & 0.20 & 0.48 & 0.03 & & 60 & 20 & 20 \\
        Demologan    & 0.60 & 0.60 & 0.62 & 0.05 & & 85 & 10 & 5 \\
        Mati         & 0.60 & 0.50 & 0.60 & 0.05 & & 30 & 50 & 20 \\
        \bottomrule
    \end{tabular}
    }
    \vspace{1ex}
    
    \raggedright
    \footnotesize
    \textit{Note: Agent behavior parameters correspond to Theory of Planned Behavior (TPB) weights: $w_a$ (Attitude), $w_{sn}$ (Social Norms), $C_e$ (Perceived Cost of Effort), and $\gamma$ (Compliance Decay Rate). Budget allocation percentages reflect the legislative constraints under the Status Quo scenario.}
\end{table}

\section{Model Calibration and Data Reconciliation}

Model calibration was executed to align the simulation with the Municipal Environment and Natural Resources Office (MENRO) audit, reconciling significant discrepancies between self-reported data and ground-level reality. While barangay interviews suggested an aggregate compliance between 40\% and 50\%, the MENRO audit established a functional segregation rate of only 10\% to 15\%. This divergence highlighted a ``Social Desirability Bias'' and an ``Intention-Action Gap,'' where officials overstated compliance or high awareness failed to translate into practice. To establish a valid ``Status Quo'' baseline of approximately 12.5\%, the model treated the low reported compliance of Poblacion (2\%) as an accurate ``Anchor Point'' while drastically reducing inflated reports from other localities, such as Binuni and Liangan East, to filter out transient project-based spikes and theoretical awareness.

\begin{table}[htbp]
    \centering
    \caption{Comparison of Acquired Compliance Data vs. Calibrated Simulation Initialization}
    \label{tab:calibration_data}
    \small
    \begin{tabular}{l c c}
        \toprule
        {Barangay} & {Acquired Data} & {Calibrated $S_0$} \\
        & \textit{(Self-Reported)} & \textit{(Simulation Start)} \\
        \midrule
        Brgy Babalaya     & 100\% & 14\% \\
        Brgy Binuni       & 95\%  & 15\% \\
        Brgy Mati         & 70\%  & 11\% \\
        Brgy Liangan East & 65\%  & 14\% \\
        Brgy Demologan    & 60\%  & 11\% \\
        Brgy Ezperanza    & 20\%  & 14\% \\
        Brgy Poblacion    & 2\%   & 2\%  \\
        \bottomrule
    \end{tabular}
    \vspace{1ex}
    
    \raggedright
    \footnotesize
    \textit{Note: ``Acquired Data'' refers to unverified compliance rates reported during Key Informant Interviews (Appendix B). ``Calibrated $S_0$'' refers to the initialized compliance state in the Agent-Based Model, tuned to match the aggregate municipal segregation rate of $\approx$12\% verified by MENRO.}
\end{table}

\subsection{Comparative Results of Policy Strategies}

The simulation experiments revealed a stark divergence in efficacy between static, equal-distribution strategies and the dynamic resource allocation discovered by the AI agent. [cite_start]The results unequivocally demonstrated that without the ``Sequential Saturation'' logic, no amount of static rebalancing between Education, Enforcement, or Incentives was sufficient to resolve municipality-wide non-compliance[cite: 31].

\begin{figure}[htbp]
    \centering
    \begin{minipage}{0.48\textwidth}
        \centering
        \includegraphics[width=\textwidth]{Baseline.jpg}
        \caption*{A. Status Quo (Baseline)}
    \end{minipage}\hfill
    \begin{minipage}{0.48\textwidth}
        \centering
        \includegraphics[width=\textwidth]{pure_enforcement.jpg}
        \caption*{B. Pure Enforcement}
    \end{minipage}
    \vspace{0.5cm}
    \begin{minipage}{0.48\textwidth}
        \centering
        \includegraphics[width=\textwidth]{pure_incentives.jpg}
        \caption*{C. Pure Incentives}
    \end{minipage}\hfill
    \begin{minipage}{0.48\textwidth}
        \centering
        \includegraphics[width=\textwidth]{ppo_result.png}
        \caption*{D. HuDRL (Sequential Saturation)}
    \end{minipage}
    \caption{Comparative compliance evolution across policy regimes. Static strategies (A-C) fail in high-density areas due to resource dilution, while the AI-driven strategy (D) achieves universal compliance.}
    \label{fig:comparative_quadrant}
\end{figure}

[cite_start]The **Status Quo (Baseline)** scenario, which distributed funds equally (\textpeso 53,571/barangay), functioned as a ``dilution mechanism''[cite: 39]. [cite_start]While small, cohesive communities like Mati and Babalaya achieved 100\% compliance through social momentum, high-density critical zones like Poblacion stagnated at 0.33\%[cite: 35, 37]. [cite_start]This validated the \textit{Urban Resource Trap} hypothesis, where anonymity and density dissolve the social pressure required for low-intensity interventions to work[cite: 38].

Single-instrument strategies performed even worse due to the economics of dilution. [cite_start]The **Pure Enforcement** strategy resulted in a collapse to 13.9\% global compliance[cite: 75]. [cite_start]The data showed that diluted enforcement was effectively useless; without reaching a critical threshold of intensity, the perceived risk of non-compliance in Poblacion remained zero[cite: 46, 47]. [cite_start]Similarly, the **Pure Incentives** strategy plateaued at 34.3\%, as the fixed budget spread across large populations resulted in microscopic rewards that failed to offset the effort cost of segregation[cite: 53, 54].

[cite_start]In contrast, the **Heuristic-Guided Deep Reinforcement Learning (HuDRL)** agent outperformed all manual strategies, achieving a terminal Global Compliance Rate of \textbf{92.8\%}[cite: 73].

\begin{table}[ht]
    \centering
    \caption{Comparative Global Compliance Rates across Policy Regimes ($Q_1$ - $Q_{12}$)}
    \label{tab:policy_comparison}
    \small
    \renewcommand{\arraystretch}{1.2}
    \begin{tabular}{l c c c c}
        \toprule
        \textbf{Policy Regime} & \textbf{Q1 (Initial)} & \textbf{Q4 (Year 1)} & \textbf{Q8 (Year 2)} & \textbf{Q12 (Terminal)} \\
        \midrule
        Status Quo (Baseline) & 10.7\% & 52.4\% & 55.3\% & 57.1\% \\
        Pure Incentives & 10.7\% & 34.6\% & 34.2\% & 34.3\% \\
        Pure Enforcement & 10.7\% & 14.9\% & 14.0\% & 13.9\% \\
        \textbf{Adaptive (HuDRL)} & \textbf{10.7\%} & \textbf{45.3\%} & \textbf{72.2\%} & \textbf{92.8\%} \\
        \bottomrule
    \end{tabular}
    \vspace{0.2cm}
    \begin{minipage}{0.9\textwidth}
        \footnotesize
        [cite_start]\textit{Note: Global Compliance is calculated as the unweighted arithmetic mean of the segregation rates of the seven constituent barangays[cite: 78]. [cite_start]The Adaptive Strategy demonstrates a delayed but exponential learning curve, surpassing the Status Quo by Quarter 6[cite: 79].}
    \end{minipage}
\end{table}

[cite_start]The superior performance of the AI was attributed to its discovery of **Sequential Saturation**—a ``King of the Hill'' logic where the agent concentrated 51\% to 69\% of the total municipal budget onto single barangays sequentially[cite: 61, 62]. [cite_start]By rotating this ``siege'' tactic from Demologan (Q2) to Poblacion (Q8-12), the agent ensured that intervention intensity always exceeded the resistance threshold of the target demographic before moving on[cite: 62, 68]. [cite_start]This validated the hypothesis that resolving the \textit{Urban Resource Trap} requires prioritizing sequential efficacy over simultaneous fairness[cite: 71].

\section{Global Sensitivity Analysis}

To evaluate the structural validity and predictive reliability of the developed Agent-Based Model (ABM), a Global Sensitivity Analysis (GSA) using Sobol indices was executed. [cite_start]Unlike local sensitivity analysis, GSA accounted for the non-linear interactions between variables across the entire multi-dimensional input space[cite: 92]. This stress-testing phase involved varying the five core behavioral parameters ($w_a, w_{sn}, w_{pbc}, c_{\text{effort}}, \text{decay}$) by $\pm 20\%$ to quantify their individual contributions ($S_1$) to the variance in global compliance rates.

\begin{figure}[htbp]
    \centering
    \includegraphics[width=0.8\textwidth]{Sobol.png}
    \caption{Global Sensitivity Analysis Results illustrating First-Order Sobol Indices ($S_1$) for behavioral parameters.}
    \label{fig:sobol_analysis}
\end{figure}

The analysis revealed that the \textbf{Cost of Effort ($c_{\text{effort}}$)} was the overwhelming driver of the model's output, accounting for approximately \textbf{83\%} of the variance in household compliance. [cite_start]This result provided computational evidence for the ``Convenience Hypothesis,'' confirming that physical friction and logistical difficulty were the primary inhibitors of waste segregation, often overriding an individual's pro-environmental intent[cite: 94, 95]. [cite_start]This aligns with observations by \citet{Yazawa2025}, who noted that structural deficiencies—such as lack of accessible collection points—often dictate policy failure regardless of public awareness[cite: 96, 97].

The secondary driver, \textbf{Social Norms ($w_{sn}$)}, acted as a critical force multiplier. The sensitivity index indicated that individual decisions were heavily contingent on the perceived compliance of the neighborhood. [cite_start]This validated the ``Cultural Inertia'' mechanism, where social pressure could either lock a community into non-compliance or, once a threshold is reached, accelerate a ``tipping point'' toward collective segregation[cite: 100, 101].

A significant revelation was the near-zero sensitivity of the \textbf{Attitude ($w_a$)} parameter. While traditional policy often prioritized IEC campaigns to change mindsets, the model demonstrated that ``environmental awareness'' alone was insufficient to drive behavior change when systemic barriers remained high. [cite_start]This quantified the ``Value-Action Gap'' documented by \citet{Zhao2022}, supporting the conclusion that extrinsic motivators (enforcement and incentives) are significantly more effective than intrinsic motivators (education) in bridging the gap between what citizens value and how they actually behave[cite: 104, 106].