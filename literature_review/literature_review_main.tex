\subsection{Solid Waste Management in the Philippines}
The national framework for waste management is governed by the Ecological Solid Waste Management Act of 2000 (R.A. 9003), which mandates source segregation, recycling, and the establishment of Materials Recovery Facilities (MRFs). However, implementation remains sub-optimal across various Local Government Units (LGUs), necessitating computational approaches to bridge the gap between national policy and local practice.

\subsubsection{Systemic and Budgetary Constraints on LGUs}
The primary challenge in implementing R.A. 9003 is the significant operational and financial burden placed on LGUs as the chief implementers. Solid waste management constitutes a high financial drain on municipal budgets, a situation compounded by a scarcity of compliant sanitary landfills and chronic underfunding for local initiatives. These deficiencies often lead to systemic institutional failures that weaken the effectiveness of the law. Consequently, overcoming these constraints requires significant political will and initiative from local officials to substantially improve municipal performance.

\subsubsection{Behavioral Drivers and Resident Non-Compliance}
Understanding the behavioral drivers of residents is essential for effective policy design. Non-compliance is frequently driven by structural friction—specifically, frustration with irregular collection services—rather than a simple lack of awareness. In some communities, minor environmental offenses have become normalized, requiring targeted interventions. While educational campaigns such as the SURWEM project raise awareness, they do not guarantee sustained behavioral change without accompanying structural support. Therefore, building trust through participatory governance and community involvement is vital for increasing compliance rates.

\subsubsection{Heterogeneity and Policy Equity}
Policy design must account for the socio-economic diversity of the population to ensure justice. Purely financial penalties, such as fines, are known to be regressive as they disproportionately affect low-income groups. To address this, reinforcement learning agents must optimize for both cost-efficiency and policy equity. This requires simulating agents with varying sensitivities to incentives and penalties based on their distinct socio-economic profiles.

\subsection{Household Behavior and Theory of Planned Behavior}

\subsection{Agent-Based Modeling as a Markov Decision Process}

\subsection{Heuristic-Guided Deep Reinforcement Learning}

\subsection{Research Gap}