%!TEX root = ../cs_conference_paper.tex

\subsection{Solid Waste Management in the Philippines}

The national framework for solid waste management is anchored by the Ecological Solid Waste Management Act of 2000 (R.A. 9003), which mandates source segregation and the establishment of Materials Recovery Facilities (MRFs) \cite{RA9003, Santos2025}. However, Local Government Units (LGUs) frequently struggle with sub-optimal implementation due to severe operational and budgetary constraints, chronic underfunding, and a lack of compliant infrastructure \cite{Camarillo2021, Coracero2021, Dalugdog2021}. Compounding these systemic failures are behavioral challenges at the household level; resident non-compliance is often driven by structural friction—such as irregular collection services—rather than a simple lack of awareness \cite{Paigalan2025, Villanueva2021}. This indicates that educational campaigns alone are insufficient without robust structural support and community trust \cite{Collado2024, Salsabila2021}. Furthermore, policy design must account for socio-economic heterogeneity to ensure justice, as purely punitive measures like fines disproportionately burden low-income groups \cite{Carpio2025, Jimenez2025}. Consequently, bridging the gap between national policy and local practice requires computational approaches that optimize for both cost-efficiency and policy equity by modeling the varying incentive and penalty sensitivities across diverse populations \cite{MedinaMijangos2020, Torkayesh2021}.

\subsection{Policy Behavioral Interventions in Waste Management}

To effectively enforce waste segregation, Local Government Units (LGUs) must implement a strategic, budget-constrained policy mix of economic incentives, regulatory penalties, and educational campaigns \cite{Dhanshyam2021, Fontaine2024}. Economic and regulatory levers, specifically hybrid reward-penalty schemes, are powerful drivers of compliance because they directly alter a household's financial cost-benefit analysis \cite{Chen2023, MuZhang2021}. However, the impact of these financial policies varies significantly across different socio-economic profiles—such as low-income households being more sensitive to fines \cite{Zhao2022}. Therefore, optimization must account for household heterogeneity to balance policy effectiveness with cost-efficiency and avoid overspending on diminishing returns \cite{Wang2023, Cheng2022}. To ensure long-term program sustainability, these financial tools must be complemented by educational and behavioral interventions \cite{Vorobeva2022}. Strategies like educational campaigns and Nudge Theory directly target the core constructs of the Theory of Planned Behavior by improving resident Attitude, Perceived Behavioral Control, and Subjective Norms \cite{LoanBalanay2023, Trushna2024}. Because these "soft" behavioral factors are critical for sustained participation, an integrated Agent-Based Modeling (ABM) framework is ideally suited to simulate and predict the complex, community-level responses to this interconnected hybrid policy mix \cite{Ceschi2021, Ma2023}.

\subsection{From Behavioral Initiation to Habitual Persistence}

Designing effective waste segregation policies requires a computational model grounded in established psychological theory. In this study, the Agent-Based Model (ABM) simulates household decision-making using the Theory of Planned Behavior (TPB), determining an individual's intention to segregate based on Attitude ($A$), Subjective Norms ($SN$), and Perceived Behavioral Control ($PBC$) \cite{Taraghi2025, Meng2018}. This cognitive process is mathematically operationalized through a linear utility function:

\begin{equation}
    U_{\mathrm{segregate}} = (w_A A + w_{SN}SN + w_{PBC}PBC) + \epsilon
    \label{eq:tpb}
\end{equation}

where the $\epsilon$ term accounts for the inherent stochasticity, or ``noise,'' in human decision-making \cite{Subedi2025}. Moving beyond individual initiation, the model incorporates non-linear ``critical mass'' dynamics. Research indicates that a committed minority reaching a tipping point of approximately 25\% can trigger a rapid cascade of adoption across the majority, implying that resource-constrained local governments can focus interventions to reach this threshold rather than funding the entire population indefinitely \cite{Centola2018, Nyborg2024}. Finally, to simulate resistance to behavioral decay once financial incentives are removed, the model integrates ``Cultural Inertia''. When waste segregation becomes a deeply established social habit, the social cost of deviating (such as peer pressure) exceeds the physical effort required, ensuring the behavior remains self-sustaining \cite{Corcoran2020, Andre2021, Farrow2020}.

\subsection{Agent-Based Modeling as a Markov Decision Process}

Agent-Based Modeling (ABM) functions as a computational "virtual laboratory" uniquely equipped to analyze complex socio-environmental systems, such as municipal solid waste management, where system-wide compliance emerges directly from micro-level household decisions \cite{Brugiere2022, TianReview2024}. Unlike aggregate methodologies such as System Dynamics that focus on macro-level stocks and flows \cite{Dhanshyam2021}, ABM is essential for this research because it accurately captures population heterogeneity and adaptive individual behavior \cite{deSouza2021, TianCase2024}. By representing households as autonomous agents with distinct socio-demographic profiles governed by the Theory of Planned Behavior (TPB), the model can simulate highly realistic, non-linear responses to dynamic policy changes \cite{Liao2024}. Furthermore, ABM's flexible architecture allows for the integration of advanced computational techniques, including machine learning classifiers, which enhances the behavioral realism of the agents and significantly improves the accuracy of predicted policy outcomes \cite{Mousavi2021, Bire2025}.

\subsection{Heuristic-Guided Deep Reinforcement Learning}

While Agent-Based Modeling (ABM) provides the simulation environment, Deep Reinforcement Learning (DRL) is essential for autonomously discovering optimal, budget-constrained policies by mapping complex, high-dimensional household states to precise municipal decisions \cite{ZhengAI2022, Rajesh2025}. To achieve this, the ABM is mathematically framed as a Markov Decision Process (MDP) that acts as a high-fidelity data generator, providing the necessary state representations and reward signals to train the DRL agent \cite{Kompella2020}. However, applying DRL to public governance introduces the "Sparse Reward Problem," as delayed causal effects in human behavior make it difficult for standard agents to learn effective strategies without failing or randomly wandering \cite{Vecerik2018, Amodei2016}. To overcome this inefficiency, the framework integrates Heuristic-Guided Reinforcement Learning (HuRL) to prune the action space and direct the agent's attention to relevant variables \cite{Cheng2021}, alongside Potential-Based Reward Shaping to provide immediate, threshold-based feedback that prevents the agent from spreading resources too thinly \cite{Ng1999}. Ultimately, these advanced techniques address a significant gap in current literature: while DRL is extensively utilized for physical, industrial waste sorting and logistics \cite{Duhayyim2022, Khan2024, HaMinh2025}, its application in optimizing governance resources to actively influence human behavior remains critically under-explored \cite{Hertweck2023, Mousavi2021Overview}.

\subsection{Research Gap}

A comprehensive review of the literature reveals a critical implementation deficit in the Philippine Solid Waste Management system under R.A. 9003, driven by weak enforcement, chronic municipal budget constraints, and a persistent gap between public awareness and actual compliance \cite{Yazawa2025, ApostolJamoralin2024}. Addressing these heterogeneous behavioral challenges requires a paradigm shift toward smart, predictive management that blends hybrid reward-penalty schemes with non-monetary educational levers \cite{Olawade2024, Udayakumar2023}. However, a critical research gap exists at the intersection of Agent-Based Modeling (ABM) and Deep Reinforcement Learning (DRL) for adaptive public policy \cite{Jimenez2025}. While existing ABM research relies on static scenario testing and DRL studies focus primarily on technical or logistical optimization, no study has developed an integrated framework where a resource-constrained Local Government Unit (LGU) autonomously optimizes the dynamic allocation of funds across enforcement, incentives, and education. To bridge this gap, this study proposes a dynamic structural modification to the Theory of Planned Behavior (TPB) by explicitly integrating an external policy term ($C_{\mathrm{Net}}$) and time-variant weights:

\begin{equation}
\begin{split}
    U_{\mathrm{segregate}} &= (w_A(t) A) + (w_{SN}(t) SN_{\mathrm{local}}) \\
    &\quad + (w_{PBC}(t) PBC_{\mathrm{infra}}) - C_{\mathrm{Net}} + \epsilon
\end{split}
\label{eq:utility_function_gap}
\end{equation}

This novel architecture simulates non-linear behavioral evolution—such as "public forgetting" and "psychological reactance"—while strategically decoupling the agent's internal psychological state from external policy levers. By transforming the TPB into a computational interface for a DRL agent, this coupled ABM-DRL framework elevates the LGU from a static administrator to a "strategic learner," providing a mathematically optimized, cost-effective decision-support tool to maximize long-term household segregation compliance \cite{Mousavi2021, Hertweck2023Values}.