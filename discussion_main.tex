%!TEX root = ../cs_conference_paper.tex

\subsection{The Determinants of Compliance}
The findings from the evaluation phase provided a critical computational justification for the strategic trajectory of the Deep Reinforcement Learning (DRL) agent, specifically its preference for the ``Sequential Saturation'' strategy over the existing education-centric ``Status Quo.'' By examining the behavioral sensitivity of the model, this discussion bridges the gap between the algorithmic outputs of the AI and the socioeconomic realities of waste management in the Municipality of Bacolod \cite{Jimenez2025}.

\subsubsection{The ``Convenience Barrier'' and the Dominance of Cost}
The overwhelming sensitivity of the model to the Cost of Effort ($c_{\text{effort}}$), which accounted for approximately 83\% of the variance in global compliance rates ($S_1 \approx 0.83$), provided mathematical validation for what is known in environmental sociology as the ``Convenience Hypothesis.'' Within the framework of this study, ``Cost'' was not defined solely by financial penalties or fines; it represented the total ``friction'' of the activity, including temporal, physical, and cognitive effort \cite{Corcoran2020}.

In the context of Bacolod, Lanao del Norte, this sensitivity implied that if the segregation process was characterized by high friction—such as the requirement to purchase specific color-coded liners or the need to traverse significant distances to collection points—household agents would default to non-compliance regardless of their pro-environmental beliefs. This finding was strongly aligned with the empirical work of \cite{Yazawa2025} and \cite{Villanueva2021}, whose research suggested that structural deficiencies often overrode individual intent.

Consequently, the DRL agent’s aggressive funding of Enforcement (to increase the cost of non-compliance) and Incentives (to offset the cost of compliance) represented a rational response to this ``Convenience Barrier.'' The AI learned that it could not simply ``educate'' away the physical friction of the waste system; it had to fundamentally alter the individual utility calculus \cite{Chen2023}.

\subsubsection{The ``Value-Action Gap'': Why Education Exhibited Diminishing Returns}
A significant revelation of the sensitivity analysis was the near-zero sensitivity of the Attitude ($w_a$) parameter. While traditional municipal policies often focused heavily on IEC campaigns, the model demonstrated that awareness alone had a low ceiling of effectiveness in the absence of structural support. This phenomenon was a computational realization of the ``Value-Action Gap'' \cite{Liao2024}.

As noted by \cite{Zhao2022} in their research on waste sorting in developing nations, extrinsic motivators such as incentives and regulatory pressure were significantly more effective than intrinsic motivators (education) in promoting consistent household participation. For the Municipality of Bacolod, these results suggested a state of diminishing returns on education. Residents likely understood the ecological importance of segregation, but the current policy framework failed to address the gap where that knowledge should turn into action. The DRL agent's decision to pivot away from IEC funding was therefore a strategic recognition that the most effective lever for immediate compliance lay in structural and extrinsic motivators \cite{Badua2022}.

\subsubsection{Cultural Inertia and the ``Tipping Point''}
The significant sensitivity to Social Norms ($w_{sn}, S_1 \approx 0.35$) validated the ``King of the Hill'' or ``Sequential Saturation'' strategy favored by the AI. Social Norms functioned as a non-linear behavioral multiplier within the social fabric of the barangays. The mechanism operated on the principle of communal visibility: when compliance was low (e.g., $<10\%$), the social norm reinforced non-compliance as the acceptable communal standard, creating a state of ``Cultural Inertia'' that resisted change \cite{Andre2021}.

However, the DRL agent identified a critical threshold—a ``Tipping Point''—where social influence flipped from a barrier to a facilitator. Once enforcement and incentives pushed a barangay’s compliance past a critical mass (approximately 70\%), the Social Norm parameter began to act as a reinforcement mechanism \cite{Centola2018}. In this state, the pressure to conform to the now-visible majority behavior sustained high compliance even as the agent reallocated resources to other areas. This explained the ``Graduation Effect'' observed in the longitudinal data.