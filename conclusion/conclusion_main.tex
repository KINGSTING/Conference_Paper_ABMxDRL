\subsection{Summary of Findings}
The evaluation of the Agent-Based Model through Global Sensitivity Analysis (GSA) and behavioral determinant mapping provided a robust computational foundation for the DRL agent's optimized policies. By transitioning from a ``Status Quo'' approach to a ``Sequential Saturation'' strategy, the model demonstrated how municipal resources could be leveraged more effectively by targeting the underlying mechanics of household decision-making \cite{ZhengAI2022}.

The primary discovery of this evaluation was the dominance of convenience as a behavioral driver. With the Cost of Effort ($c_{\text{effort}}$) accounting for 83\% of output variance, the study confirmed that logistical friction was the single greatest inhibitor of waste segregation in the Municipality of Bacolod \cite{Yazawa2025}. Policies that did not actively reduce this ``cost'' were statistically likely to fail regardless of the level of public support.

Furthermore, the results highlighted a distinct ``Value-Action Gap'' characterized by the near-zero sensitivity of the Attitude ($w_a$) parameter. This mathematically illustrated that while IEC campaigns were foundational, they had reached a point of diminishing returns \cite{Zhao2022}. The DRL agent's strategic shift toward extrinsic motivators was thus justified; awareness alone could not bridge the gap between intention and action when systemic barriers remained high.

Finally, the model identified social multipliers as the key to long-term sustainability. The significant influence of Social Norms ($w_{sn} \approx 0.35$) provided the mechanism for the ``Graduation Effect,'' allowing for the eventual reallocation of funds without a collapse in segregation rates \cite{Andre2021}. In conclusion, the DRL agent’s preference for localized enforcement and incentives was a sophisticated response to the socio-technical barriers in Philippine waste management, offering a data-driven roadmap toward a resilient compliance ecosystem \cite{Jimenez2025}.