\begin{abstract}
The implementation of the Ecological Solid Waste Management Act (RA 9003) remains a critical challenge for Local Government Units (LGUs) in the Philippines. Compliance is often hindered by budget constraints, lack of behavioral data, and the complexity of enforcing segregation at the household level. This study presents a novel policy optimization framework combining Agent-Based Modeling (ABM) with Deep Reinforcement Learning (DRL). We model the Municipality of Bacolod, Lanao del Norte, as a dynamic environment where household agents react to policy interventions based on the Theory of Planned Behavior (TPB). 
A custom Heuristic-guided Deep Reinforcement Learning (HuDRL) agent acts as the LGU policymaker, learning to maximize segregation compliance under strict budget constraints. Our simulations reveal that the HuDRL agent outperforms traditional "Status Quo" policies and standard PPO agents by discovering a "Sequential Saturation" strategy—focusing resources on specific zones to build social norms before expanding. Furthermore, Global Sensitivity Analysis using Sobol Indices identifies "Cost of Effort" as the primary driver of non-compliance, suggesting that LGUs should prioritize logistical support over purely punitive measures.
\end{abstract}

\begin{IEEEkeywords}
Agent-Based Modeling, Deep Reinforcement Learning, Waste Management Policy, Smart Governance, Sobol Sensitivity Analysis, Theory of Planned Behavior
\end{IEEEkeywords}